%% Copernicus Publications Manuscript Preparation Template for LaTeX Submissions
%% ---------------------------------
%% This template should be used for copernicus.cls
%% The class file and some style files are bundled in the Copernicus Latex Package, which can be downloaded from the different journal webpages.
%% For further assistance please contact Copernicus Publications at: production@copernicus.org
%% https://publications.copernicus.org/for_authors/manuscript_preparation.html


%% Please use the following documentclass and journal abbreviations for discussion papers and final revised papers.

%% 2-column papers and discussion papers
\documentclass[acp, manuscript]{copernicus}



%% Journal abbreviations (please use the same for discussion papers and final revised papers)


% Advances in Geosciences (adgeo)
% Advances in Radio Science (ars)
% Advances in Science and Research (asr)
% Advances in Statistical Climatology, Meteorology and Oceanography (ascmo)
% Annales Geophysicae (angeo)
% Archives Animal Breeding (aab)
% ASTRA Proceedings (ap)
% Atmospheric Chemistry and Physics (acp)
% Atmospheric Measurement Techniques (amt)
% Biogeosciences (bg)
% Climate of the Past (cp)
% DEUQUA Special Publications (deuquasp)
% Drinking Water Engineering and Science (dwes)
% Earth Surface Dynamics (esurf)
% Earth System Dynamics (esd)
% Earth System Science Data (essd)
% E&G Quaternary Science Journal (egqsj)
% European Journal of Mineralogy (ejm)
% Fossil Record (fr)
% Geochronology (gchron)
% Geographica Helvetica (gh)
% Geoscience Communication (gc)
% Geoscientific Instrumentation, Methods and Data Systems (gi)
% Geoscientific Model Development (gmd)
% History of Geo- and Space Sciences (hgss)
% Hydrology and Earth System Sciences (hess)
% Journal of Micropalaeontology (jm)
% Journal of Sensors and Sensor Systems (jsss)
% Mechanical Sciences (ms)
% Natural Hazards and Earth System Sciences (nhess)
% Nonlinear Processes in Geophysics (npg)
% Ocean Science (os)
% Primate Biology (pb)
% Proceedings of the International Association of Hydrological Sciences (piahs)
% Scientific Drilling (sd)
% SOIL (soil)
% Solid Earth (se)
% The Cryosphere (tc)
% Weather and Climate Dynamics (wcd)
% Web Ecology (we)
% Wind Energy Science (wes)


%% \usepackage commands included in the copernicus.cls:
%\usepackage[german, english]{babel}
\usepackage{tabularx}
%\usepackage{cancel}
\usepackage{multirow}
%\usepackage{supertabular}
%\usepackage{algorithmic}
%\usepackage{algorithm}
%\usepackage{amsthm}
%\usepackage{float}
%\usepackage{subfig}
\usepackage{rotating}

\usepackage{booktabs}
%\usepackage{ltablex}

\begin{document}

\title{Evaluation of climate model aerosol trends with ground-based observations over the last two decades - an AeroCom and CMIP6 analysis}


% \Author[affil]{given_name}{surname}
\Author[1]{Augustin}{Mortier}
\Author[1]{Jonas}{Gli{\ss}}
\Author[1]{Michael}{Schulz}
%alphabetical after here
\Author[2]{Wenche}{Aas}
\Author[3]{Elisabeth}{Andrews}
\Author[4]{Huisheng}{Bian}
\Author[5]{Mian}{Chin}
\Author[6]{Paul}{Ginoux}
\Author[7]{Jenny}{Hand}
\Author[5]{Brent}{Holben}
\Author[8]{Zhang}{Hua}
\Author[9]{Zak}{Kipling}
\Author[1]{Alf}{Kirkevåg}
\Author[10]{Paolo}{Laj}
\Author[11]{Thibault}{Lurton}
\Author[12]{Gunnar}{Myhre}
\Author[13]{David}{Neubauer}
\Author[1]{Dirk}{Olivié}
\Author[14]{Knut}{von Salzen}
\Author[12]{Ragnhild}{Skeie}
\Author[15]{Toshihiko}{Takemura}
\Author[16]{Simone}{Tilmes}


\affil[1]{Norwegian Meteorological Institute, Oslo, Norway}
\affil[2]{NILU, Norwegian Institute for Air Research, Kjeller, Norway}
\affil[3]{Cooperative Institute for Research in Environmental Sciences,University of Colorado, Boulder, Colorado, USA}
\affil[4]{Maryland Univ. Baltimore County (UMBC), Baltimore, MD, USA}
\affil[5]{NASA Goddard Space Flight Center, Greenbelt, Maryland, USA}
\affil[6]{NOAA, Geophysical Fluid Dynamics Laboratory, Princeton, NJ, USA}
\affil[7]{Cooperative Institute for Research in the Atmosphere, Colorado State University, Fort Collins, CO, USA}
\affil[8]{Laboratory for Climate Studies, National Climate Center,
China Meteorological Administration, Beijing, China}
\affil[9]{European Centre for Medium-Range Weather Forecasts, Reading, UK}
\affil[10]{Univ. Grenoble Alpes, CNRS, IRD, Grenoble INP, Institute for Geosciences and Environmental Research, Grenoble, France}
\affil[11]{Met Office Hadley Centre, Exeter, UK}
\affil[12]{CICERO Center for International Climate and Environmental Research, Oslo, Norway}
\affil[13]{Institute for Atmospheric and Climate Science, ETH Zurich, Zurich, Switzerland}
\affil[14]{Environment Canada, Montréal, Canada}
\affil[15]{Research Institute for Applied Mechanics, Kyushu University, 6-1 Kasuga-koen, Kasuga, Fukuoka, Japan}
\affil[16]{National Center for Atmospheric Research (NCAR), Boulder, Colorado, USA}





%% The [] brackets identify the author with the corresponding affiliation. 1, 2, 3, etc. should be inserted.
%% If an author is deceased, please add a further affiliation and mark the respective author name(s) with a dagger, e.g. "\Author[2,$\dag$]{Anton}{Aman}" with the affiliations "\affil[2]{University of ...}" and "\affil[$\dag$]{deceased, 1 July 2019}"


\correspondence{Augustin Mortier (augustinm@met.no)}

\runningtitle{Aerosol Trends}

\runningauthor{Augustin Mortier}

\received{}
\pubdiscuss{} %% only important for two-stage journals
\revised{}
\accepted{}
\published{}
%% These dates will be inserted by Copernicus Publications during the typesetting process.


\firstpage{1}

\maketitle


%\tableofcontents

\begin{abstract}
 This study presents a multi-parameter analysis of aerosol trends over the last two decades at regional and global scales. Regional time series have been computed for a set of nine optical, chemical composition and mass aerosol properties by using the observations from several ground-based networks. From these regional time series the aerosol trends have been derived for different regions of the world. Most of the properties related to aerosol loading exhibit negative trends, both at the surface and in the total atmospheric column. Significant decreases of aerosol optical depth (AOD) are found in Europe, North America, South America, North Africa and Asia, ranging from -1.2\%/yr to -3.1\%/yr. An error and representativity analysis of the spatially and temporally limited observational data has been performed using model data subsets in order to investigate how likely the observed trends represent the actual trends happening in the regions over the full study period from 2000 to 2014. This analysis reveals that significant uncertainty is associated with some of the regional trends due to time and space sampling deficiencies. The set of observed regional trends has then been used for the evaluation of 10 models (6 AeroCom Phase III models, and 4 CMIP6 models) and the CAMS-reanalysis dataset, and their skills in reproducing the aerosol trends. Model performance is found to vary depending on the parameters and the regions of the world. The models tend to capture trends in AOD, column \r{A}ngstr\"{o}m exponent, sulfate and particulate matter well (except in North Africa), but show larger discrepancies for coarse mode AOD. The rather good agreement of the trends, across different aerosol parameters between models and observations, when co-locating them in time and space, implies that global model trends, including those in poorly monitored regions, are likely correct. The models can help to provide a global picture of the aerosol trends by filling the gaps in regions not covered by observations. The calculation of aerosol trends at a global scale reveals a different picture from that depicted by solely relying on ground based observations. Using a model with complete diagnostics (NorESM2) we find a global increase of AOD of about 0.2\%/yr between 2000 and 2014, primarily caused by an increase in the loads of organic aerosol, sulfate and black carbon.
\end{abstract}


\copyrightstatement{TEXT}


\introduction  %% \introduction[modified heading if necessary]
As one of the key gears involved in the climate mechanism \citep{poschl2005atmospheric}, and as a predominant component of air quality that affects human health \citep{burnett2014integrated}, aerosols have been increasingly subject to observation over the last two decades, both from ground and space-based platforms \citep{holben2001emerging,kaufman2002satellite}.  Aerosols are also recognized to have an important role for the fertilization of the Amazon forest \citep{yu2015fertilizing}, and in other socioeconomic fields such as the solar energy production \citep{Li11867,labordena2018blue,Sweerts2019}.

Through their direct, semi-direct and indirect effects \citep{rap2013natural,johnson2004semi,lohmann2005global}, aerosol particles are crucial for the estimation of the radiative forcing. Currently, the overall estimate of aerosol radiative forcing is associated with high uncertainties \citep{haywood2000estimates, stocker2014climate}. Some of the reasons for these uncertainties reside in the heterogeneity of atmospheric particles, both in terms of their microphysical and optical properties, as well as the high variability of these aerosols in space and time. Different regions of the world exhibit contrasting aerosol properties \citep{holben2001emerging}, which can vary depending on the seasons, from year to year, and possibly exhibit inter-annual trends \citep{streets2009anthropogenic}. In addition to natural emissions such as sea salt and dust, anthropogenic sources of aerosol add another layer of complexity. The Second Industrial Revolution, which relied on the use of fossil fuel energy, has had a significant impact on the aerosol load on a global scale, and on the local air quality, resulting in severe pollution episodes, such as the famous 1952 smog event in London \citep{bell2004retrospective} that caused the death of thousands of people within a few days.
Starting in the 1970s in the US, and in the 1990s in Europe, mitigation measures were implemented to limit the emission of particles and other pollutants \citep{bryner1995blue,turnock2016impact} resulting in significant improvements in terms of air quality and particle concentration levels \citep{likens2001long}. In recent decades there has been a shift of anthropogenic emissions from Europe and North America to the developing nations, which are now facing, in varying degrees, the air quality issues that affected Europe and North America 40 years ago \citep{streets2008aerosol,ramachandran2012aerosol}.

In order to provide realistic radiative forcing estimates and projections, it is important for the atmospheric models to be able to capture the long-term aerosol trends caused by both natural and anthropogenic variations.

%The AeroCom initiative focuses on the assessment of the modelling of the aerosols. Specific experiments are conducted within AeroCom, both for the assessment of general aerosol modelling, but also with a focus on the individual aerosol species, such as dust \citep{Huneeus-acp-2011} or organic aerosols \citep{Tsigaridis-acp-2014}. Dedicated control experiments aim to enable an assessment of the global aerosol modelling and to capture changes and improvements over time. Related studies focus, for instance, on aerosol forcing (\citep{schulz2006radiative}), the aerosol life-cycle (\citep{textor2006} or on aerosol optical properties (\citep{kinne-acp-2006,jonaseval}).

Assessing and improving the modelling of aerosols in global earth system models is the main objective of the AeroCom-project. Specific experiments are conducted within this initiative with a focus on individual aerosol species, such as dust \citep{Huneeus-acp-2011} or organic aerosols \citep{Tsigaridis-acp-2014}, while dedicated control experiments aim to enable an assessment of the global aerosol modelling. Both \cite{kinne-acp-2006} and, more recently, \cite{jonaseval} present evaluations of global aerosol optical properties simulations by focusing on AeroCom control experiment data for a specific year.

This study presents an overview of the aerosol trends for multiple aerosol parameters (optical and chemical) over the last two decades using ground based observation network data as a reference for the evaluation of the models' skills in reproducing those trends.

To serve that purpose, this study addresses the following two questions:
\begin{itemize}
    \item What are the observed aerosol trends over the last two decades in the different regions of the world? (Section \ref{obs_trends})
    \item Can the climate models reproduce these observed trends? (Section \ref{mod_evaluation})
\end{itemize}

Then, having developed an understanding of the models' skills in reproducing the observed aerosol trends, the last section of this study aims to answer the following question:
\begin{itemize}
    \item What are the global aerosol trends derived from the model data? (Section \ref{global_trends})
\end{itemize}

The CAMS reanalysis dataset and output from six AeroCom models and four CMIP6 models (both model groups performed historical experiments) are evaluated in this work. CMIP6 (Coupled Model Intercomparison Project, Phase 6) is an intercomparison project organised by the WCRP (World Climate Research Program). Participating models will contribute to the assessment of the climate change in the upcoming 2024 IPCC (Intergovernmental Panel on Climate Change) report.

Figure \ref{fig:hist_runs} presents the time series of modeled global AOD (aerosol optical depth) between 1850 and 2014. All of the climate models appear to exhibit a large increase in AOD, especially between 1950 and 1990 \citep{tegen-jgr-2000}, followed by more stable conditions up to the present. While the models show some diversity in absolute values, the trends (focus of this paper) seem to be consistent among the different models at a global scale. Long-term  monitoring of many optical and chemical parameters were initiated in the late 1990's \citep[e.g.,][]{holben2001emerging,laj-2019-499}, providing a high quality dataset for the investigation of aerosol trends over the last two decades \citep[e.g.,][]{collaudcoen-2019,HAND2019117025,aas2019global}.  These observational datasets also offer an opportunity to validate the modeled trends in this period. Since 2014 is the last year available from the CMIP6 historical runs, we focus this study on the aerosol trends in the 2000-2014 period.


\section{Datasets}
A set of nine column and in situ surface aerosol datasets are used in this study. The observation networks and the models providing output for these parameters are reported in Table \ref{table:datasets}.


\subsection{Observations}

For each of the parameters used in this study, data of the highest quality level provided by the different observation networks were used. Mountain sites, corresponding to an elevation above 1000 m, were excluded, mainly because global models have problems simulating the aerosol distribution in complex terrain \citep{kinne2013mac}.

\subsubsection{Columnar aerosol optical properties}

The AErosol RObotic NETwork (AERONET) is a network established by NASA (National Aeronautics and Space Administration), and expanded by national and international collaborations. AERONET operates aerosol ground-based measurements in the different regions of the world \citep{holben2001emerging}. The observation of the columnar aerosol properties is performed by standardized and calibrated solar-powered CIMEL Electronique sunphotometers. These instruments measure the solar radiation reaching the surface of the Earth at different wavelengths and for different optical geometries. A new version of the sunphotometer (CE318-T) is also able to perform night-time measurements using the moon as light-source \citep{barreto2016new}. The direct measurements (aiming at the light-source) allow for the derivation of the aerosol optical depth (AOD), and the Ångström exponent (AE) which are related to the amount and size of the particles, respectively. The spectral information can be further utilized to derive the AOD for the fine and the coarse particles, split by diameter less than or greater than 1.2 \unit{µm} \citep{o2003spectral}. Three different data quality levels are available depending on the application of cloud filtering and correction for instruments calibration derivations \citep{smirnov2000cloud,smirnov2004aeronet}. The level 2.0 version 3 daily data, which provides automatic instrument anomaly quality controls \citep{giles2019advancements}, are used in this study for four different parameters:
AOD (calculated at 550 nm), AE (calculated using 440 nm and 870 nm channels), \chem{AOD_{f}} (or fine AOD), and \chem{AOD_{c}} (or coarse AOD) corresponding to the AOD of the particles whose diameter is less than and greater than 1.2 \unit{µm}, respectively.

\subsubsection{Particulate matter concentrations}
The particulate matter (PM) measurements are from EMEP (covering Europe), and IMPROVE (for North America). The PM data have been made available either via the EBAS database infrastructure  (\url{http://ebas.nilu.no}), or in the original IMPROVE data to be found in the VIEWS database (\url{http://views.cira.colostate.edu/}). Both \chem{PM_{10}} and \chem{PM_{2.5}} (with unit \unit{µg\,m^{-3}}) are used in this study. Note that the \chem{PM_{10}} size fraction of particles below 10 \unit{µm} encompasses the \chem{PM_{2.5}} aerosol mass below 2.5 \unit{µm}.

The first PM measurements in EMEP started in 1996 and the number of sites increased steadily in the following decade \citep{torseth2012}. Most of the sites use the gravimetric method for both size fractions, though some used automated monitors, i.e. TEOM FDMS or b-attenuation. The EMEP monitoring complies with the European standards, i.e EN12341:2014 for the gravimetric methods and  EN16450:2017 for the automatic methods.

The IMPROVE network has been operating since 1988 at predominantly remote and rural sites across the United States. This ensures a good representativity of the measurements as some chemical species contributing to PM observations (i.e., Organic carbon) can exhibit different seasonality and spatial variability. IMPROVE uses four separate modules to collect samples for speciated \chem{PM_{2.5}} analysis and gravimetric \chem{PM_{2.5}} and \chem{PM_{10}} bulk mass measurements. Samples are collected every third day for 24 h and reported at local conditions. \chem{PM_{2.5}} and \chem{PM_{10}} mass concentrations are determined from Teflon filters from two separate modules sampling with \chem{PM_{2.5}} and \chem{PM_{10}} inlets, respectively. The gravimetric mass measurements are not performed at controlled relative humidity and temperature, and a laboratory relocation in 2011 resulted in unstable weighing conditions. Therefore, gravimetric mass measurements from 2011-2018 were subject to potentially high relative humidity conditions and likely contain particle bound water on the filters that could bias trends \citep{Hand2019}.

\subsubsection{Sulphate aerosol concentration}
 The sulphate aerosol concentration (\chem{SO_4}) dataset is a subset of the global data presented in \cite{aas2019global} and is based on measurements obtained in different regional networks as described in Table \ref{table:datasets}. The sulfate aerosol concentrations are obtained from analysis of aerosol filters. In the EMEP, CASTNET, CAPMON and EANET networks, these are either sampled with a \chem{PM_{10}} inlet or a total aerosol inlet, with no specific size cut off effective, using a filterpack sampler.  In the IMPROVE network, sulfate measurements are done using a filterpack sampler with a \chem{PM_{2.5}} inlet. The filters are typically analysed by ion chromatography after water extraction of the aerosol filter.

The data have been screened to be of satisfactory quality. Urban sites are not included, nor are sites where the surroundings have changed considerably in the period in question.
In \cite{aas2019global} the data  were averaged to monthly means.  When the data have lower sampling frequency than daily, samples are weighted prior to averaging in accordance with how many days were sampled in a given month.

\subsubsection{Scattering and absorption coefficients}
For the surface in-situ PM measurements, the scattering and absorption coefficients measurements were accessed through EBAS database infrastructure. The level 2 data (quality controlled, hourly averaged, reported at standard temperature and pressure (STP) conditions) were used. Detailed information on the quality assurance and quality control procedures for GAW aerosol in-situ data are available in \citep{laj-2019-499}.  The difference in measurement conditions (i.e., observations being made at STP versus models simulating at ambient conditions) was not expected to impact the calculated trends so no adjustment was made to account for this.

  Scattering and absorption coefficients are measured by different instruments:

\begin{itemize}
 \item Scattering coefficients ($\sigma_{sp}$, in \unit{Mm^{-1}}), were measured by integrating nephelometers. For better consistency in the model comparisons (model data for these parameters are reported for RH=0\%), only the measurement data obtained when the relative humidity in the instrument was lower than 40\% were utilized \citep{pandolfi2018european}.
 \item Absorption coefficients ($\sigma_{ap}$, in \unit{Mm^{-1}}), were obtained from filter-based absorption photometers.
\end{itemize}

Due to the scarcity of stations available for long-term trend analysis (only 28), the presence of regionally non-representative stations (e.g., stations located near roads, in cities), difficult to capture by global models, can have large effects on the computation of the regional average time series. The urban stations have therefore been removed from this analysis.

Altogether the same data selection procedures (exclusion of stations, removal of outliers) and corrections (conversion to coefficients at 550 nm wavelength) were applied as in \cite{jonaseval}, which describes the AeroCom evaluation of the Control 2019 experiment, analysing AeroCom simulations of the year 2010 in detail.

\subsection{Models}
A set of 10 climate and aerosol models and a aerosol reanalysis dataset are used in this study. Their main characteristics are reported in Table \ref{table:models}. These models can be separated into three main groups.

\subsubsection{CAMS-Reanalysis}
The CAMS reanalysis, which is the successor to the MACC reanalysis (Monitoring Atmospheric Composition and Climate), is the latest global reanalysis dataset of atmospheric composition produced by the Copernicus Atmosphere Monitoring Service \citep{inness2019cams}. It is produced using 4DVar data assimilation in the CY42R1 model cycle of the ECMWF (European Centre for Medium-Range Weather Forecasts) Integrated Forecast System (IFS), with 60 hybrid sigma/pressure vertical levels. The model used in the CAMS reanalysis includes several updates to the aerosol and chemistry modules on top of the standard CY42R1 release. The IFS model assimilates several satellite products, from aerosols (AOD) to greenhouse gases (CO2, CH4) \citep{inness2019cams}, where most relevant for aerosol trends are data from both MODIS sensors and AATSR/ATSR2. Daily data, from the ECMWF data archive (MARS), were used in this study. The CAMS reanalysis data set covers the period January 2003 to near real time. The three first years of this study period (2000-2002) are missing for this model.

\subsubsection{AeroCom phase III}
Initiated in 2000, the AeroCom-project (https://aerocom.met.no) is an open international initiative of scientists interested in the advancement of the understanding of the global aerosol and its impact on climate \citep{schulz2006radiative}. Different model experiments have been conducted during the third phase of this project, started in 2015, in order to investigate specific topics (e.g., dust, volcanic aerosols, aerosol absorption, hygroscopicity, etc.) but also the general modelling of the aerosols (control experiments). The model versions and parametrizations used in AeroCom are closely linked to the versions used for CMIP6 and, for instance, AerChemMIP (Aerosol Chemistry Model Intercomparison Project) climate experiments.

In this study, we use the model outputs from the historical AeroCom experiment, whose main aim is to understand the regional trends in aerosol distribution from 1850 to 2015 and to quantify the aerosol forcing with a main emphasis on the direct aerosol effect. The models were also run in various configurations, providing different degrees of constraints on the evolving meteorological conditions, such as using monthly fixed sea-surface temperature (SST), historically evolving SSTs, and basic meteorology fields, e.g., wind for a given year.

\subsubsection{CMIP6}
The upcoming 2024 IPCC sixth assessment report (AR6) will feature new state-of-the-art CMIP6 models with model runs in higher resolution and with new physical processes. An overview of the experimental design and organisation can be found in \cite{eyring2016overview}.  In this study, we use a preliminary extract of the data of four CMIP6 models from the historical experiment, as available on ESGF nodes (Earth System Grid Federation: https://esgf.llnl.gov), which provided output from 1850 to 2014. 2014 was selected as the last year of the study period of the analysis presented here.



\section{Methods}

\subsection{Regional time series}
Due to the nature of the processes involved in the emission and the deposition of aerosols, one can expect different trends in different regions of the world. Instead of investigating the trends obtained at each individual observation station in a given region, we resort here to the analysis of average regional time series as computed by assembling all measurements at stations in each region. One advantage of this method is that a single trend can be computed in a given region, with an associated significance and uncertainty. It is difficult, apart from a diversity analysis, to define such an uncertainty when combining individual trends. Also, with our aggregation method, even a station that has not provided a sufficient amount of data for computing a trend at its location can still contribute to the computation of a regional time series. The computation of such aggregated regional time series makes most sense in regions exhibiting similar seasonal patterns.

\subsubsection{Regions definition and observations coverage}
Seven regions are considered in this study. The definition of these regions has been done in a pragmatic way to limit the number of geographic areas investigated, but altogether also provides a global coverage when considering the ensemble of all regions. The Americas and Africa have been separated into northern and southern sections. In order to assemble the sites most affected by Saharan dust, the North Africa region has been extended to the north beyond the continent itself. Stations located in the south of Spain, Cyprus and Greece contribute to the regional time series in the region we are calling North Africa. The regions coordinates can be found in the supplementary info.

As seen in Figure \ref{fig:map_obs}, the regions do not have a similar coverage in terms of observations. North America and Europe have the highest concentrations of instruments monitoring aerosol trends.
\begin{itemize}
 \item AERONET is the most important network in terms of number of instruments. More than 1000 observation points, with more or less long time series, are found across the globe. The highest density of instruments is in Europe and in the central part of North America (US). The lowest densities are found in southern Africa and Australia.
 \item Particulate Matter: 227 instruments are used in this study and are spread mostly over Europe and North America.
 \item \chem{SO_4}: Altogether 346 instruments have been operating, mostly in North America and Europe. A few stations are also located in Asia and North Africa.
 \item $\sigma_{sp}$ and $\sigma_{ap}$: Combined for both parameters approximately 50 stations are spread over North America, Europe, North Africa and Asia. Due to time coverage issues (2005 is the first year where in situ optical data are available in the European time series), the data from 2000 up to the year 2018 were used to compute the regional time series of these two parameters.
\end{itemize}


\subsubsection{Time series aggregation requirements}
The regional time series are computed by combining, for each month, the valid data of all the stations in the corresponding region. In order to construct consistent and robust regional time series, some additional criteria are required to be met to provide a valid point (a station with valid measurements) going into the regional time series. Stations with very short time series (e.g AERONET DRAGON campaign stations) are eliminated by requiring a minimum of 300 valid daily measurements in the whole period from 2000 to 2014, which reduces, as an illustration, the number of AERONET stations from 1015 to 437. A minimum of three valid stations is required to be present in the overall regional time series to produce a valid point. In other words, if the available time resolution is daily, at least three stations need to provide valid data for a certain day in order to produce a valid regional mean for that day.  The list of the station names contributing to the computation of the regional time series can be found in the supplementary info.

When all criteria are fulfilled for a given month in the regional time series, the median and the first and third quartiles are computed from all valid data points available. The quartiles provide an indication of the intra-regional variability. An example of regional time-series are shown in Figure \ref{fig:ts_aod} for AOD.


\subsection{Trends calculation}

\subsubsection{Yearly, regional time series}
For all of the parameters, the trends are computed based on the yearly averages of the regional time series. Using the yearly averages eliminates any issues caused by the seasonal cycles (observed for most of the aerosol parameters used in this study) during the calculation of the trend slope. In order to ensure the statistical robustness of these yearly averages, the time averaging is performed step-by-step with specific time constraints. By starting at the finest time resolution available in the data, monthly, seasonal and then yearly averages are computed when the following criteria are fulfilled:
\begin{itemize}
 \item at least 5 days per month are available (when daily observations are available).
 \item at least 1 month with data per season is present (seasons defined as JFM, AMJ, JAS, OND).
 \item all 4 seasons are available for a given year.
\end{itemize}
These temporal constraints offer a reasonable compromise between the availability and robustness of the yearly statistics.

\subsubsection{Trends computation}
We use the same methodology as described by \cite{aas2019global} to derive the trends of the regional time series. The significance of the trends is tested with the Mann-Kendall test \citep{hamed1998modified}. The related p-value is used to determine if the trend is significant or not within a confidence interval of 95\%. The slope is calculated with the Theil-Sen estimator which is less sensitive to outliers than standard least-squares methods \citep{sen1968estimates}. At least least 7 valid yearly regional averages (50\% of time coverage) are required in the regional time series for the computation of a slope.

An uncertainty is provided for each trend by combining the error of the slope calculation itself to the error of the residuals:

\begin{equation}
 Uncertainty = \sqrt{{\left (\frac{\Delta m}{y(2000)}\right )}^{2} + {\left ( \frac{m \cdot \Delta r}{y(2000)^2}\right )}^{2} }
\end{equation}

where $\Delta m$ is the Theil-Sen estimator 95\% confidence interval, $y(2000)$ is the value of the regression line at the year 2000, $m$ is the value of the Theil-Sen slope and $\Delta r$ is the averaged error on the residuals computed based on the difference between the linear trend and the yearly mean values of the regional time series.

The trend is provided as a relative trend (\%/yr) with respect to the first year of the time period (2000).

\subsection{Representativity of the trends}
The number of available points used to compute the regional time series is not constant in time. For a given observation station, the number of points available might vary in time due to the nature of the measurements. For instance, classic sunphotometers only measure in the daytime and in cloud free conditions. Due to seasonal daylight and cloud condition variations, clear seasonal cycles are observed in the number of observations of AOD. The density of the different observation networks can also change with time. The early development of the different observation networks usually coincided with an increase in the number of observation stations. More recently, primarily for funding reasons, some networks have reduced the number of stations. This variation in the number of available measurements raises the question of time representativity for the computation of the trends.

Associated with this time representativity issue comes the space representativity issue. The data coverage is uneven across the different regions. Moreover, within a single region, the observation stations might be located in contrasting environments. Stations located in environments that are more urban, or rural, or mostly affected by natural particles, might have trends differing from the trend associated with the whole region.

Some studies have focused on the representativity of the observation stations by investigating the biases of different optical properties \citep{wang2017,schutgens2017spatio,schutgens2019site}. The analysis here is dedicated to characterising the  representativity of the observation networks specifically for the purpose of computing the trends. These two perspectives on representativity might give different results, since a station associated with a bias could still have a representative tendency in time. In order to evaluate the effect of the partial space and time sampling of the observations for the evaluation of the trends, two sensitivity studies, focusing on the time sampling and the space sampling, have been conducted using NorESM2 model data subsets. For each of these studies, the trends are computed for one reference ($Ref$) and one experiment ($Exp$) dataset, and compared with each other. The reference dataset corresponds to the model data co-located to the available observations while the experiment dataset uses all model points.
\begin{itemize}
 \item Time representativity study
       \begin{itemize}
        \item $Ref_{time}$: Model data collocated in space and time with available observations
        \item $Exp_{time}$: Model data collocated in space with available observations using the complete model time-series
       \end{itemize}
 \item Space representativity study
       \begin{itemize}
        \item $Ref_{space}$: Model data collocated in space with available observations using the complete model time-series (=$Exp_{time}$)
        \item $Exp_{space}$: All of the model grid-points in the region using the complete model time-series
       \end{itemize}
\end{itemize}

The difference between the relative trends are computed for each parameter and region. In order to summarize the representativity, those differences are then converted into a score (\unit{\%}) by using a mapping function which has been defined based on a normal distribution. The choice of the parameters describing this function leads to a representativity score of 100\% when there is no difference in the trends computed for a reference and an experiment dataset, while a difference of 0.5\unit{\%/yr} obtained with these two datasets would indicate a representativity score of 50\%. Finally, the total score is computed as the mean of the time and the space representativities.

An example of the calculation is presented in Figure \ref{fig:representativity} for AOD in Europe and North America. In both regions, the $Ref_{time}$ dataset, corresponding to the available observations, reveals strong seasonal cycles when considering the number of points used to compute the regional time-series. These cycles are observed with most of the sunphotometer datasets since the instruments only operate during daytime and cloud free conditions, and the amount of daylight and clouds varies with the season. Together with this seasonal cycle, one observes an increase in the number of points with time, which reflects the increasing number of stations over these two regions.

The trends in Europe show similar values for the time study, which means that the trend is not greatly affected by the variation of the available measurements in time. The difference is larger when considering all the grid-boxes of the domain, but the overall difference of the two studies corresponds to a representativity of 76\%. In North America, the difference in the three trends is larger, particularly for the space study trend. This means that the trend obtained in the whole region is significantly different from the trend obtained when considering only the grid points where observation stations are located. It should be mentioned that the ocean grid-points are not filtered out when computing the trends over the whole domain. For this reason, the regions containing a greater proportion of ocean grid-points, where the trends are most likely to differ from those observed over land, will tend to have a lower spatial representativity, such as North America.

This representativity study illustrates that the partial coverage in time and space of the observations leads, in some cases, to artificial trends. The representativity scores are discussed for each parameter in the following section together with the trend estimate results.

\section{Results}

\subsection{Trends in observations}\label{obs_trends}
This section presents the trends in the observations computed for the different parameters and over the predefined regions. In order to compare the trends observed for the set of nine aerosol parameters in a consistent manner, we focus on the relative trends, with the reference set to the year 2000, as the first year of the study period. The means for the year 2000, reported in Table \ref{table:obs_2000mean}, reveal a large inter-regional variability.


The AOD is more than three times higher in Asia (AOD=0.37) than in North America and Australia (AOD=0.10). Intermediate AOD values are found in Europe and South Africa, while the second highest load is found in North Africa (AOD=0.26). In most regions, the AOD is largely dominated by its fine mode fraction (\chem{AOD_{f}}), but this is not the case in North Africa (or Australia), where the persistent presence of desert dust makes the coarse mode (\chem{AOD_{c}}) contribution to the total AOD similar in size to the fine mode contribution. This predominance of coarse particles is reflected in the AE values which exhibit lower values in North Africa (AE=0.70) and Australia (AE=1.00).

The PM observations are primarily available from Europe and North America. \chem{PM_{10}} observations are also available in the North Africa region as defined in this analysis, but these stations are  located in the northern part of the region, i.e., in southern Europe, which is less affected by the dust sources than the AERONET stations, which cover the whole region including the surrounding deserts. Both \chem{PM_{10}} and \chem{PM_{2.5}} are larger in Europe than in North America, with different relative proportions. In Europe, \chem{PM_{2.5}} represent 76\% of the \chem{PM_{10}}, as compared to only 57\% in North America. This difference in the relative proportion of fine particles against coarse particles in Europe and North America may be due in part to our definition of regions. \cite{PUTAUD20101308} presented a phenomenology of PM data in Europe showing coarse aerosol tended to be highest in southern Europe which in our study is part of the North Africa region. The discrepancy in the relative proportions of coarse and fine aerosol in Europe and North America may be exacerbated by both a decrease in North America of the fine particles concentration due to pollution mitigation strategies coupled with the growth of the coarse mass due to increasing contributions of natural and agricultural sources, particularly in the western half of the U.S. \citep{HAND2019117025}.

\chem{SO_4} means (surface mass concentrations) for the year 2000 range between 1.45 and 2.98 \unit{µg\,m^{-3}} with the lowest value occurring in North America and the highest value in North Africa (sites in southern Europe). Similar means are found in Europe and Asia, around 2 \unit{µg\,m^{-3}}, though one should bear in mind that there are relatively few sites in Asia and they are not located in the most polluted areas in China and India \citep{aas2019global}.

Analogous to the surface \chem{PM_{10}} measurements, $\sigma_{sp}$ is higher in Europe (34 \unit{Mm^{-1}}) than in North America (23 \unit{Mm^{-1}}). The same feature is found for $\sigma_{ap}$ which also has higher values in Europe than North America.



The relative trends for the 2000-2014 period are shown in Figure \ref{fig:obs_trends}. The heatmap is dominated by blue color, which indicates mostly negative trends, especially when considering the parameters related to aerosol burden (i.e., the extensive parameters). Usually, the lowest p-values (<0.05) are associated with the lowest uncertainties not shown in the same figure though. The largest circles (highest significance of trend) are more confidently associated with a decrease/increase of the aerosol property in the time period 2000-2014 since the value of the trend is greater than the uncertainty. The uncertainties are presented in Figure \ref{fig:bars}. Among the 38 computed trends, 22 are associated with a representativity score higher than 50\% and 24 are significant at a 95\% confidence level.

\begin{itemize}
 \item In Europe, both columnar and surface parameters reveal statistically significant decreases. With the exception of $\sigma_{ap}$, for which the associated uncertainty of the trend exceeds the trend itself, the trends computed for other parameters are associated with uncertainties lower than the trends values. A decrease in AOD (-2.8\%/yr) is found for both fine and coarse mode particles. This is consistent with the negative trends found at some individual stations in this region \citep{glantz2019}. The fine mode is decreasing more than the coarse mode, which is consistent with the decrease observed for AE. The same shift in aerosol size is found at the surface since \chem{PM_{2.5}} has decreased by a factor of two relative to \chem{PM_{10}}. These trends could result from the mitigation measures aimed at reduced anthropogenic aerosol emissions. This is more directly observed in the decrease of \chem{SO_4} (-1.5\%/yr). We find a somewhat lower trend than what was reported in \cite{aas2019global} (-2.67\%/yr), but that could be explained by the differences in the methodology (trends computed from the regional time series, in this study, against a statistical average of the trends computed at the individual stations) and/or the definition of the region. The stations in the Mediterranean Basin, where a larger decrease is found (-4.3\%/yr), are attributed to the North African region in this study.

 The representativity study reveals that the observed trends are actually representative for the whole period and region for all of the parameters. A good agreement is found with the trends obtained at individual stations reported by \cite{collaudcoen-2019}, who found decreases of -2.92\%/yr for $\sigma_{sp}$ and -4.2\%/yr for $\sigma_{ap}$, as compared to -2.5\%/yr and -2.0\%/yr in this study.
 \item In North America, similar trends are found for the columnar properties as were found for Europe. AOD is decreasing at a rate of 1.3\%/yr, a 55\% percent smaller trend than observed in Europe, but the North America reference value in 2000 is 40\% lower than the reference value in Europe. One can note that the representativity scores are higher for AE than for AOD, while these two parameters have the same amount of data. This means that the trends in AE are probably more homogeneous in space and time, which makes the same amount of available observations more representative in the case of AE. The decreases observed for both \chem{PM_{2.5}} (-2.0\%/yr) and \chem{PM_{10}} -1.2\%/yr are significant and in the same range of values as the trends found in Europe. However, the actual trends for \chem{PM_{10}} and \chem{PM_{2.5}} are probably somewhat more negative than found here. The possible bias is caused by increased relative humidity during weighing, resulting in more particle bound water and thus higher mass, after the relocation of the laboratory in 2011. \cite{Hand2019} reported that the  decrease in \chem{PM_{2.5}} from 2005 through 2016 was -2.6\%/yr, while it was -3.9\%/yr for the reconstructed fine mass correcting for the possible bias in the measurements. \chem{SO_4} decreases by about 3\%/yr, which is twice as large as the decrease observed in Europe, where the reference value is however larger than in North America. The sulfate trend is similar to the trend reported by \cite{aas2019global} in this region (-3.15\%/yr). The regional time series extend farther back in time for $\sigma_{sp}$ and $\sigma_{ap}$ in North America than in Europe. However, no significant trends are found for these data sets. This is in contrast to \cite{collaudcoen-2019} which found a large decrease for $\sigma_{sp}$ (-2.66\%/yr). Our study used regionally averaged time series to calculate the trend rather than regionally averaged trends as was done by \cite{collaudcoen-2019}. This probably illustrates the difference of methodology which consists of computing the mean of station trends in one case, and the trend of a regional time series in the other case, especially when only few measurements are available. However, as shown by the representativity study (\ref{fig:obs_trends}), the non-significant increase of +0.0\%/yr found, in this study, with the observations is similar to the trend derived over the whole region and using complete time series of the NorESM2 model data.
 Similar values are found in this study and by \cite{collaudcoen-2019} for $\sigma_{ap}$ (-1.85\%/yr) although the trend is, here, not significant. The IMPROVE network also measures filter absorption using a Hybrid Integrating Plate and Sphere (HIPS) system \citep{Warren2016}. These data are not included in this study, but \cite{Warren2016} reports a significant decrease (-2.7\%/y) in the light absorption coefficients from 2005 to 2015.
 \item All of the columnar properties show significant decreasing trends in South America, except for \chem{AOD_{c}}. As shown in the regional time series in Figure~\ref{fig:ts_aod}, the observed decrease in AOD coincides with a global diminution of the intensity of the seasonal peaks happening around September and resulting from the Amazonian forest fires \citep{aragao201821st}. These peaks are highly variable from year to year and could greatly affect the trend when considering another time period. With a rate of -2.0\%/yr, the largest decrease of AE is found in this region. While no significant trend is found for \chem{AOD_{c}}, the tendency towards increasing coarse particles is probably due to the production of local dust as a result of the increasing deforestation \citep{werth2002local,betts2008effects}.
 \item In North Africa, while significant decreases are found for all AOD parameters, an increase of AE (+1.2\%/yr) is observed, which indicates an increase in the proportion of fine particles with time. This is consistent when considering the AOD of the fine and coarse modes, which reveal a larger decrease for \chem{AOD_{c}}. \cite{chin2014multi} also found a decrease in dust in the Sahara/Sahel in the time period 1980-2009 due to reduced 10m-wind speed, possibly caused by an increase in sea surface temperature (SST) in the North Atlantic.
 \item AE is also increasing in Asia as a combination of a (not significant) increase in \chem{AOD_{f}} and a significant decrease in \chem{AOD_{c}}. The increase in AE is likely tied to increases in anthropogenic emissions which are associated with fine mode aerosol. This result is consistent with the trend reported by \cite{yoon2012trend} at some individual stations. At the same time, we observe an increase of \chem{SO_4} of 3.8\%/yr, which is consistent with the trend reported in \cite{aas2019global}. This increase is associated with a large uncertainty ($\pm$4\%/yr ) due to a drop in the already small number of stations available in the region, especially between 2010 and 2012. Indeed, with a maximum of 12 stations, a few stations missing can greatly affect the computation of the regional time series. This is reflected by the representativity study which reveals a score lower than 40\% for this parameter.
 \item No significant trends are found in Australia, although the representativity is greater than 50\% for \chem{AOD_{f}}.

\end{itemize}

This multi-parameter trend analysis reveals a decrease in most of the parameters relating to aerosol burden, both in the total column and at the surface level. In Asia, the trends in \chem{AOD_{f}}, AE and \chem{SO_4} suggest an increase in the proportion of the finer particles. While differences might be expected when comparing regional trends with trends computed at individual stations, the trends are usually consistent with those previously reported in the literature.  \cite{DEMEIJ201275} focused on regional AOD trends in the 2000-2009 period; despite the differences in the study periods and the methodologies involved, trends consistent with those found in this study are found for most of the regions.

\subsection{Evaluation of the models trends against observations}\label{mod_evaluation}


In order to evaluate the trends from the models, the regional time series have been computed with the model output collocated in space and time to the available observations at the station level. The model trends are computed in a similar manner to the trends for the observation datasets. The results, shown in Figure \ref{fig:bars}, reveal (a) the differing abilities of the various models to reproduce of the observed trends, and (b) the model performance depends on the parameters and the regions.


\begin{itemize}
 \item AOD: the models show trends with the same sign as the observed trends over all the regions except in Asia, where the associated uncertainties are usually larger than the trend values. Some differences among the three model groups are observed when investigating the different regions:
       \begin{itemize}
        \item EUROPE: all the groups underestimate the observed decrease in AOD. With an average decrease of -1.0\%/yr, the CMIP6 models exhibit the largest underestimation, while the best performance is obtained with CAMS-Rean (-2.0\%/yr). The AP3 models' trends range from -1.3\%/yr to -2.3\%/yr.
        \item NAMERICA: in contrast to the results for EUROPE, on average, all of the models overestimate the observed AOD decrease in NAMERICA even though two models of the AP3 group simulate lower trends than are found for observations. The consistency in the trends is very high within the CMIP6 group over this region.
        \item SAMERICA: CAMS-Rean slightly overestimates the observed AOD decrease while almost all of the models in the two other groups underestimate this decrease. A few of the models simulate positive trends, but these are associated with large uncertainties.
        \item NAFRICA: all the models capture the observed decreasing AOD tendency. With a trend of -3.0\%/yr, CAMS-Rean is the closest to the observed trend (-2.8\%/yr). AP3 and CMIP6 multi-model trend averages are both equal o -2.1\%/yr.
        \item ASIA: A large inter-model variability is found in this region where the uncertainty is also important. The means of the trends of each group are close to 0\%/yr.
       \end{itemize}
 \item \chem{AOD_{f}}: usually, the same patterns are found as for AOD. The models that underestimated the AOD underestimate \chem{AOD_{f}} and vice versa. For \chem{AOD_{f}} and the following parameters, only NorESM2 provides data for the CMIP6 group.
       \begin{itemize}
        \item EUROPE: the underestimation of the decrease in \chem{AOD_{f}} captured by the models is larger than the underestimation of AOD.
        \item ASIA: as for AOD, the trends are associated to large uncertainties and show a large inter-model variability.
       \end{itemize}
 \item \chem{AOD_{c}}: the performance of the models is not as good as for \chem{AOD_{f}}. This is also observed when evaluating the models for a single year \citep{jonaseval}. The inter model variability is also higher since some models simulate \chem{AOD_{c}} trends in opposite directions in some regions.
       \begin{itemize}
        \item EUROPE: while the observations exhibit a significant decrease, CAMS-Rean and all of the AP3 models exhibit increasing values for \chem{AOD_{c}}. NorESM2 from CMIP6 simulates a decrease consistent with the observations.
        \item SAMERICA: All of the models simulate large increases, from +1.2\%/yr up to +8.5\%/yr  which are not visible in the observations (+0.1\%/yr).
        \item NAFRICA: the models reproduce the observed decrease of 3.1\%/yr to some extent (from -0.6\%/yr to -2.7\%/yr). The fact that some models with fixed SST (e.g ECHAM-HAM) reproduce this decrease does not support the hypothesis of the SST changes impacting dust emissions. \citep{fan2004impact,bauer2005impact,bauer2007sulfate,neubauer2019global}.
        \item ASIA: CAMS-Rean captures a similar negative trend as computed with the observations dataset. As with \chem{AOD_{f}}, no certain trend can be identified  in this region with the NorESM2 CMIP6 model.
       \end{itemize}
 \item AE: the trends are usually smaller than for AOD in the respective regions. This can mean that the amount of the particles is more subject to variations than the size (type) of these particles but could also illustrate that AE is less sensitive to the change in a relative sense. This feature is visible with both observations and models.
       \begin{itemize}
        \item EUROPE and NAMERICA: one model in the AP3 group (ECHAM-HAM) simulates a significant positive trend in AE while negative tendencies are found in the observation and with the other models.
        \item SAMERICA: all of the models simulate negative AE trends, most of them significant, in agreement with the observations. CAMS-Rean and the AP3 models tend to underestimate the decrease, while NorESM2 CMIP6 model tends to overestimate it.
        \item NAFRICA: CAMS-Rean does an excellent job of reproducing the observed AE increase (+1.3\%/yr versus +1.2\%/yr). The significant trends of the AP3 models range from -0.4\%/yr to +2.0\%/yr. The increase of AE supports the theory of enhanced scavenging of dust by anthropogenic aerosols.
        \item ASIA: the AP3 models and the NorESM2 CMIP6 model exhibit significant positive trends, which is also the case for the observations. CAMS-Rean does not capture any significant trend in this region.
       \end{itemize}
 \item \chem{PM_{2.5}}: Almost all the models simulate significant decreases over Europe and North America, in good agreement with the observations. The CMIP6 model performs however better in North America, while it underestimates the extent of the decrease in Europe. Further analysis reveals that, despite the fact that it does a good job reproducing the \chem{PM_{2.5}} trend in North America, CAMS-Rean exhibits a large positive bias in this region when considering the absolute values (+100\%). In North Africa, both CAMS-Rean and AP3 models capture the significant decrease seen in the observations.
 \item \chem{PM_{10}}: In North Africa, only CAMS-Rean reproduces the observed significant decrease. Positive trends are found for all the models in the AP3 (except GEOS) and CMIP6 groups. As for \chem{PM_{2.5}}, NorESM2 has better performance in North America. CAMS-Rean produces a trend twice as high as the observed trends both over Europe and North America.
 \item \chem{SO_4}: The AP3 and CMIP6 models perform quite well for the \chem{SO_4} surface concentration. The magnitude of the model trends is however higher than the observed trends in all the regions except North Africa.
 \item $\sigma_{sp}$ and $\sigma_{ap}$:  as mentioned in the previous section, the observations trends have been computed for these two parameters using data until 2018. The two models providing output for these parameters are NorESM2 and OsloCTM3. NorESM2 provides data until 2014, so the NorESM2 trends correspond to the period [2000-2014], while OsloCTM3 provides data until 2017 and the respective trends correspond to [2000-2017].
       \begin{itemize}
        \item EUROPE: a significant decrease is found in the observations for both $\sigma_{sp}$ and $\sigma_{ap}$ but this is not captured by the models for which the calculated trends are associated with large uncertainties.
        \item NAMERICA: A significant decrease of -1.3\%/yr is found with both NorESM2 and OsloCTM3 for $\sigma_{sp}$ which is not seen in the observations. For $\sigma_{ap}$, the two models capture a similar trend as derived from the observations (-1.5\%/yr).
       \end{itemize}
\end{itemize}

This model trends evaluation reveals some key points. First, CAMS-Rean, which assimilates AOD, performs the best for capturing the trends of this parameter. Second, a large inter-model variability is generally found over Asia, where the observed trends are also the most uncertain.
Considering the total column, the models usually perform rather well for AOD, \chem{AOD_{f}}, and AE, but show lower skill for \chem{AOD_{c}}. At ground level, the models perform well for both \chem{SO_4} concentration and PM. The trends in $\sigma_{sp}$ and $\sigma_{ap}$ computed from regional time series are associated with large uncertainties due to the limited number of stations. This is exacerbated by the fact that data was only available from two models for these parameters.


\subsection{Trends in models}\label{global_trends}

\subsubsection{Global trends}

As discussed previously, the regional trends found are probably not always representative of the trends in the extended regions and over the whole study period. The reasons are the partial spatial and temporal coverage of the ground based observations. Moreover, the observation stations are obviously located on land. This does not allow for a depiction of a global aerosol trends and is unfortunate as sea salt particles are among the most predominant aerosols on Earth \citep{schulz2004sea}.

Unlike observations, models provide data at a global scale and for the entire study period. The completeness of these model datasets offers the opportunity to derive global aerosol trends. In order to provide an assessment of the aerosol trends at a global scale, we present, in this section, the trends computed with NorESM2 (CMIP6 group), which provides data for all of the nine parameters considered in this study. The calculation of the global trend is made by averaging the absolute trends computed at each grid-point of the model and using all timestamps in the study period. In order to provide a relative trend, this absolute trend is normalized to the global average of the considered parameter for the year 2000. The global trends are reported for the nine aerosol parameters in Table \ref{table:global_trends}. The global maps, shown in Figure \ref{fig:global_trends}, enable investigation of the spatial variability of these trends.

While the observed trends of the three AOD parameters show a decrease in most of the regions of the world, the global AOD trend is actually positive (+0.2\%/yr). This global increase is also found with other models. Averages of the models from the CAMS-Rean and the AP3 groups simulate global trends of about +0.2\%/yr and +0.3\%/yr respectively. Within the CMIP6 group, IPSL and CESM2 also exhibit positive trends (+0.7\%/yr and +0.3\%/yr), consistent with NorESM2, while CanESM simulates a negative trend (-0.8\%/yr). The relative increase of 0.2\%/yr found with NorESM2 corresponds to an absolute rate of +0.0028/decade, which is in excellent agreement with the global trend (over the oceans) of +0.003/decade reported by \cite{zhang2010decadal} using MODIS data. The increase of AOD is observed to be larger for the fine fraction, with an increase of about +0.6\%/yr, as compared to +0.1\%/yr for \chem{AOD_{c}}. As seen in Figure \ref{fig:global_trends}, similar geographical patterns are found for the three AODs: increase in South-Africa and East-Asia and decrease in Europe and in the US. The increasing AOD observed in Canada is dominated by an increase of \chem{AOD_{f}} in this region. The prominent increase of AOD in Indonesia seems to be linked to a large increase of \chem{AOD_{c}}. The tropical Pacific Ocean, off the west coast of South America, has significant positive modelled trends in both AOD and \chem{AOD_{f}}. Almost no significant trend is found south of 60\textdegree S.

The model also simulates an increase for AE on a global scale, with a rate of +0.3\%/yr. This suggests a shift towards smaller particles. The largest increases are found over Canada, Greenland, Siberia and the Pacific Ocean. There are some distinct outliers around 60\textdegree S. In the Atlantic, we find a decrease of AE off the east coast of the US, which is consistent with the decrease of \chem{AOD_{f}} in the same region.

The trends in both \chem{PM_{2.5}} and \chem{PM_{10}} exhibit similar geographical features as are observed for AOD. In addition, one finds large and significantly increasing trends in the high Arctic that could be explained by a change in the air mass circulation pattern, or by the increase of open sea, which might contribute to a higher production of sea salt aerosols \citep{Willis2018,abbatt-2019}. The global averages show that \chem{PM_{2.5}} is increasing faster than \chem{PM_{10}}  (+0.2\%/yr vs. +0.1\%/yr), which is consistent with the increasing AE, suggesting a relatively higher fraction of fine particles with time.

The surface \chem{SO_4} concentration trends map reveals two large contrasting regions. Significant decreases are found over North America and Europe, while significant increases are found over southern and eastern Asia and southern to central parts of Africa. This illustrates the shift of polluting activities from the developed countries to the developing countries during the last two decades. With an overall increase of +0.4\%/yr, the global trend is positive.

The $\sigma_{sp}$ trends are very similar to those observed for both \chem{PM_{2.5}} and \chem{PM_{10}}. The same geographical patterns are found, as well as the global average trend which amounts to an increase of 0.2 \%/yr over the study period.

$\sigma_{ap}$ reveals increasing tendencies over most of the grid-boxes of the model, except in Europe, the eastern part of the US, and Australia. This explains why a large positive global trend is obtained for this parameter, with an average of +1.5\%/yr. Further analysis shows a good spatial correlation with the black carbon OD (Optical Depth) that exhibits a strong global positive trend of +2.3\%/yr, as discussed below.

Table \ref{table:global_trends} also contains the trends computed for the different aerosol parameters when combining only the grid-points where an observation station is located, whether measurements are available or not. Significant differences in 'global' trends can be found when observations are not provided over some regions. This is most obvious for \chem{SO_4} for which the observation stations are located mostly in Europe and North America and exhibit decreasing values, while only a few stations are located in the regions associated with increasing values. In this case, the computation of the trends by considering only observation station grid-boxes leads to a global decrease of -3.9\%/yr while consideration of all of the grid-boxes of the model leads to a global increase of +0.4\%/yr.


\subsubsection{Contribution of main aerosol species to the AOD trends}

The averaged global trend computed by NorESM2 indicates an increase of AOD in the 2000-2014 period with a rate of about 0.2\%/yr. The trends in AE, \chem{AOD_{f}} and \chem{AOD_{c}} indicate that fine mode particles are primarily responsible for this increase in the atmospheric column.

In this section, we investigate the trends of the major aerosol species simulated by NorESM2. For that purpose, the absolute trends of the individual contribution of these species to the AOD were computed, as well as the trends in the loads and the emissions. The trends of OD and loads are shown in Figure \ref{fig:species}. In this version, NorESM2 simulates a large proportion of sea salt. This is the result of a model tuning used for reaching climate equilibrium. While the model attributes too much OD to SS, the trends should not be affected by this tuning.

The relative increase of AOD of +0.2\%/yr corresponds to an absolute increase in AOD of +3.1e\unit{^{-4}/yr}. This positive trend is dominated by an increase in the species specific ODs of the organic aerosols (OA), \chem{SO_4} and black carbon (BC), which are responsible for an increase of the OD of about +2.0e\unit{^{-4}/yr}, +0.7e\unit{^{-4}/yr} and +0.4e\unit{^{-4}/yr}, respectively. The relative OD trends give a different ranking since the highest increase is found for BC (+2.5\%/yr), followed by OA (+0.5 \%/yr). On average, the trends for dust and sea salt OD are slightly negative (-0.1e\unit{^{-4}/yr}). Note - these species trends include any associated water which can change as function of relative humidity.

The trends in OD do not necessarily represent the trends in the aerosol loads which do not include associated water. The different species have different global mass extinction coefficients (calculated in this study as OD/load, dust: 1.8 \unit{m^{2} g^{-1}}, SS: 4.3 \unit{m^{2} g^{-1}}, OA: 5.6 \unit{m^{2} g^{-1}}, \chem{SO_4}: 5.3 \unit{m^{2} g^{-1}}, BC: 7.6 \unit{m^{2} g^{-1}}). For sea salt, opposite trends are observed for the sea salt OD (positive trend) and the sea salt load (negative trend). The analysis of the global maps (not shown in this study) reveals that the largest increases of the sea salt load  happen in Indonesia and near the North Pole and result in a relatively larger increase of OD in these areas. These localized increases in sea salt OD drive the global sea salt OD trend and are due, at least in part, to the higher relative humidity at these latitudes which makes the sea salt, which is very hygroscopic, more efficient at light extinction.


\conclusions  %% \conclusions[modified heading if necessary]

The main findings of this multi parameter trends analysis are listed below:
\begin{itemize}
 \item The observations exhibit mostly negative trends regarding the extensive parameters in the different regions of the world. Significant decreases are found in Europe, North America, South America, North Africa and Asia. In Asia, AE is increasing in time and is consistent with increases in \chem{AOD_{f}} and \chem{SO_4}, which reflects the regional increase of the anthropogenic aerosols in that region in the overall study period from 2000 to 2014.
 \item Some observation networks allow for the derivation of representative trends over the whole study period. In other cases, the limited temporal and spatial coverage of the observations can induce artificial and/or highly uncertain trends when using regional time series. Among the 38 computed trends with observation data, 22 are considered as representative of the actual trends occurring in the whole region and study period.
 \item The models tend to capture observed AOD, AE, \chem{SO_4} and PM trends but show larger discrepancies regarding \chem{AOD_{c}}. The smaller amount of data available for establishing $\sigma_{sp}$ and $\sigma_{ap}$ trends makes the validation of the modeled trends more uncertain.
 \item The rather good agreement of the trends, across different aerosol parameters between models and observations, when co-locating them in time and space, implies that global model trends, including those in poorly monitored regions, are likely correct.
 \item The global trends computed with NorESM2 (CMIP6 group) model data give a different picture than the trends obtained when using only ground-based observations. Global positive trends are found for all of the parameters related to aerosol loading. The trends in AOD are dominated by the increase of the fine particles both in the column and at the surface. This tendency toward finer particles is consistent with the positive trend in AE. This increase appears to be dominated by organic aerosol, for which the emissions have increased in the study period, and by \chem{SO_4} aerosol whose sources were shifted from Europe and North America to Africa and East Asia where a global positive \chem{SO_4} trend is found.
\end{itemize}

Some elements were not considered in this study which could be investigated in order to complete the aerosol trends picture:
\begin{itemize}
 \item Some regions are associated with strong seasonal cycles. In South America, the regional time series shows high peaks in AOD, associated with forest fires in the late summer, whose intensity greatly varies from year to year. In Africa, a strong seasonal contrast is also found due to the transport of desert dust at altitude in the summer months \citep{mortier2016, ogunjobi2008synoptic}. The computation of the seasonal trends would allow characterization of the tendencies in such extreme or synoptic aerosol events.
 \item  This study shows that the trends computed from the ground-based observations networks are not representative of the global aerosol trends due to the inhomogeneities in data spatial coverage. The satellites providing a global Earth observation could be utilized for the evaluation of the model trends in the regions lacking observations and over the oceans \citep{hsu2012global,zhang2010decadal}.
 \item The trends in the meteorological parameters could be investigated in parallel with the aerosol trends because they affect the aerosol life cycle and their optical properties \citep{che2019large}. Hypothetical trends in wind velocity could produce trends in the loads of sea salt and dust and, as seen in the last section, trends in OD could also be enhanced by relative humidity changes. Changes in temperature could impact the magnitude of the biogenic emissions. Indeed, increasing temperatures, associated with changes in land use and high atmospheric \chem{CO_2} concentrations have been shown to lead to an increase of the BVOC emissions \citep{penuelas2010bvocs}. Finally, trends in precipitation that are responsible for aerosol wet scavenging would directly impact trends in aerosol loads.
 \item Several studies have linked the trends in anthropogenic aerosols to radiative forcing variations while investigating sources of global dimming and brightening \citep{streets2006two,norris2007trends}. It could be of interest to evaluate how much the modeled trends deviations, as compared to the observations, are affecting the calculation of the radiative forcing, in the different regions of the world, and at a global scale.
 \item While the mountain sites were excluded from this study, it could be of interest to investigate the trends at higher altitude (which may be related to changes in long range transport) by including the in situ and remote sensing stations higher than 1000 m (Jungfraujoch, Mauna Loa Observatory, etc.). Similarly, it may also be of interest to look at trends in smaller regions (e.g., split North America into several sub-regions which are more internally consistent in terms of climate and environment than the large NAMERICA region defined here, or consider southern Europe as its own region rather than combining it with the North Africa region as was done here).
\end{itemize}


%% The following commands are for the statements about the availability of data sets and/or software code corresponding to the manuscript.
%% It is strongly recommended to make use of these sections in case data sets and/or software code have been part of your research the article is based on.

\codeavailability{The observation and model data were read and collocated with the pyaerocom python library (\url{https://github.com/metno/pyaerocom}, version 0.10.0).} %% use this section when having only software code available

%\dataavailability{TEXT} %% use this section when having only data sets available


%\codedataavailability{TEXT} %% use this section when having data sets and software code available


%\sampleavailability{TEXT} %% use this section when having geoscientific samples available


%\videosupplement{TEXT} %% use this section when having video supplements available


%\appendix
%\section{} %\label{sec:representativity} %% Appendix A

%\subsection{}     %% Appendix A1, A2, etc.


%\noappendix       %% use this to mark the end of the appendix section

%% Regarding figures and tables in appendices, the following two options are possible depending on your general handling of figures and tables in the manuscript environment:

%% Option 1: If you sorted all figures and tables into the sections of the text, please also sort the appendix figures and appendix tables into the respective appendix sections.
%% They will be correctly named automatically.

%% Option 2: If you put all figures after the reference list, please insert appendix tables and figures after the normal tables and figures.
%% To rename them correctly to A1, A2, etc., please add the following commands in front of them:


%\appendixfigures  %% needs to be added in front of appendix figures

%\appendixtables   %% needs to be added in front of appendix tables

%% Please add \clearpage between each table and/or figure. Further guidelines on figures and tables can be found below.

\authorcontribution{A. M. has coordinated the study, has been responsible for the statistical calculation and analysis and wrote the paper, J. G. is the main developer of the pyaerocom library, M. S. has provided feedback on the methods and the manuscript, W. A, E. A, J. H, and P. L have provided in situ data, contributed to the observation dataset section writing and provided feedback on the manuscript, B. H. is the PI of AERONET, H. B., M. C., P. G., Z. H., Z. K., A. K., T. L., G. M., D. N., D. O., K. S., T. T., and S. T. have provided model output data and feedback on the manuscript.} %% this section is mandatory

\competinginterests{The authors declare no competing interests.
} %% this section is mandatory even if you declare that no competing interests are present

%\disclaimer{TEXT} %% optional section

\begin{acknowledgements}
 Data providers from all the regional and global networks are greatly acknowledged for sharing and submitting their data to be used. DN acknowledges funding from the European Union’s Horizon 2020 research and innovation programme project FORCeS under grant agreement No 821205. The ECHAM-HAMMOZ model is developed by a consortium composed of ETH Zurich, Max Planck Institut für Meteorologie, Forschungszentrum Jülich, University of Oxford, the Finnish Meteorological Institute and the Leibniz Institute for Tropospheric Research, and managed by the Center for Climate Systems Modeling (C2SM) at ETH Zurich.

 The CESM project is supported primarily by the National Science Foundation (NSF). This material is based upon work supported by the National Center for Atmospheric Research, which is a major facility sponsored by the NSF under Cooperative Agreement No. 1852977. Computing and data storage resources, including the Cheyenne supercomputer (doi:10.5065/D6RX99HX), were provided by the Computational and Information Systems Laboratory (CISL) at NCAR.  All simulations were carried out on the Cheyenne high-performance computing platform https://www2.cisl.ucar.edu/user-support/acknowledging-ncarcisl, and are available to the community via the Earth System Grid.
\end{acknowledgements}




%% REFERENCES

%% The reference list is compiled as follows:

%\begin{thebibliography}{}

% \bibitem[AUTHOR(YEAR)]{LABEL1}
% REFERENCE 1

% \bibitem[AUTHOR(YEAR)]{LABEL2}
% REFERENCE 2

%\end{thebibliography}

%% Since the Copernicus LaTeX package includes the BibTeX style file copernicus.bst,
%% authors experienced with BibTeX only have to include the following two lines:
%%
\bibliographystyle{copernicus}
\bibliography{mybib.bib}
%%
%% URLs and DOIs can be entered in your BibTeX file as:
%%
%% URL = {http://www.xyz.org/~jones/idx_g.htm}
%% DOI = {10.5194/xyz}


%% LITERATURE CITATIONS
%%
%% command                        & example result
%% \citet{jones90}|               & Jones et al. (1990)
%% \citep{jones90}|               & (Jones et al., 1990)
%% \citep{jones90,jones93}|       & (Jones et al., 1990, 1993)
%% \citep[p.~32]{jones90}|        & (Jones et al., 1990, p.~32)
%% \citep[e.g.,][]{jones90}|      & (e.g., Jones et al., 1990)
%% \citep[e.g.,][p.~32]{jones90}| & (e.g., Jones et al., 1990, p.~32)
%% \citeauthor{jones90}|          & Jones et al.
%% \citeyear{jones90}|            & 1990



%%TABLES
\clearpage
\begin{table}
 \caption{List of observations and model datasets used in this study (see text for explanation).}

\footnotesize
 \begin{tabularx}{\textwidth}{lllX}
  \tophline
  Parameter   & Type    & Observation networks & Models                                                                                                    \\
  \middlehline
  AOD         & Column  & AERONET\textsuperscript{1}             & ECMWF-Rean; NorESM2; SPRINTARS; ECHAM-HAM; GEOS; OsloCTM3; GFDL-AM4; BCC-CUACE; CanESM5; CESM2; IPSL-CM6A \\
  \chem{AOD_{f}}     & Column  & AERONET             & NorESM2; SPRINTARS; ECHAM-HAM; GEOS; OsloCTM3; GFDL-AM4                                                             \\
  \chem{AOD_{c}}     & Column  & AERONET             & ECMWF-Rean; NorESM2; SPRINTARS; ECHAM-HAM; OsloCTM3; GFDL-AM4                                  \\
  AE          & Column  & AERONET             & ECMWF-Rean; NorESM2; SPRINTARS; ECHAM-HAM; GEOS; OsloCTM3; GFDL-AM4                                       \\
  \chem{PM_{2.5}}  & Surface & EMEP\textsuperscript{2}; IMPROVE\textsuperscript{3}                 & ECMWF-Rean;  NorESM2;  SPRINTARS;  ECHAM-HAM; GEOS                                                                                       \\
  \chem{PM_{10}}   & Surface & EMEP; IMPROVE                 & ECMWF-Rean; NorESM2; SPRINTARS; ECHAM-HAM; GEOS                                                           \\
  \chem{SO_4}    & Surface & EMEP; IMPROVE; CASTNET\textsuperscript{4}; CAPMoN\textsuperscript{5}; EANET\textsuperscript{6}             & ECMWF-Rean; NorESM2; SPRINTARS; ECHAM-HAM; GEOS; OsloCTM3; BCC-CUACE                                      \\
 $\sigma_{sp}$ & Surface & GAW-WDCA\textsuperscript{7} (incl. IMPROVE; NOAA-FAN\textsuperscript{8}; ACTRIS\textsuperscript{9}; EMEP)  & NorESM2; OsloCTM3                                                                                          \\
  $\sigma_{ap}$  & Surface & GAW-WDCA (incl. NOAA-FAN; ACTRIS; EMEP)  & NorESM2; OsloCTM3                                                                                        \\
  \bottomhline
 \end{tabularx}
 \belowtable{
\textsuperscript{1}Aerosol Robotic Network
 \textsuperscript{2}The European Monitoring and Evaluation Program
 \textsuperscript{3}Interagency Monitoring of Protected Visual Environments
 \textsuperscript{4}Clean Air Status and Trends Network
 \textsuperscript{5}The Canadian Air and Precipitation Monitoring Network
 \textsuperscript{6}Acid Deposition Network in East Asia
 \textsuperscript{7}Global Atmosphere Watch - World Data Centre for Aerosol
 \textsuperscript{8}National Oceanic and Atmospheric Administration Federated Aerosol Network
 \textsuperscript{9}Aerosol, Clouds, and Trace Gases Research Infrastructure
 }
 \label{table:datasets}
\end{table}

\clearpage
\begin{sidewaystable}
 \caption{Information on models used in this study (CAMS-Rean: CAMS reanalysis, AP3: AeroCom phase III; CMIP6: historical experiments from CMIP6).}
\footnotesize
 \begin{tabularx}{\textwidth}{llllllX}
  \toprule
  Model      & Group     & \begin{tabular}[c]{@{}l@{}}Natural \\ interactive emissions\end{tabular} & \begin{tabular}[c]{@{}l@{}}Anthropogenic \\ emissions\end{tabular} & Meteorology & \begin{tabular}[c]{@{}l@{}}Resolution \\ (degree)\end{tabular} & References                                                          \\ \midrule
  ECMWF-Rean & CAMS-Rean & D, SS                      & MACCity & RA & 0.7x0.7                  & \cite{inness2019cams,zhang2009asian}                                                                    \\
  SPRINTARS  & AP3       & D, SS, DMS, Oce VOC      &  C: \chem{SO_2}, BC, OC                & N           & 0.56x0.56                  & \cite{takemura2000global,takemura2002single,takemura2005simulation} \\
  ECHAM-HAM  & AP3       & D, SS, DMS                          & C: \chem{SO_2}, BC, OC                          & fSST           & 1.875x1.875                  &  \cite{tegen2019global,neubauer2019global}                                       \\
  GEOS       & AP3       & D, SS, DMS, Oce VOC    & O: \chem{SO_2}, \chem{SO_4}, BC, OC, \chem{NH_3}     & N           & 1.00x1.00                  &       \cite{bian2017investigation,chin2002tropospheric,colarco2010online}                   \\
  OsloCTM3   & AP3       & D, SS                         & C:                         & S           & 2.25x2.25                  & \cite{lund2018concentrations,myhre2009modelled}                     \\
  GFDL-AM4   & AP3 & D, SS, DMS, Oce\&Veg OC, & C: \chem{SO_2}, \chem{SO_4}, BC, OC & fSST\&N  & 1.x1.25 &  \cite{zhao2018agfdl,zhao2018bgfdl}                 \\
  BCC-CUACE  & AP3       & D, SS, DMS                          &  C:\chem{SO_2}, BC, OC                         & F           & 2.8x2.8                  & \cite{zhang2012simulation,zhang2014application,wang2014improvement}                                                                    \\
  NorESM2    & CMIP6     & D, SS, DMS, MSA, BVOC      & C: \chem{SO_2}, \chem{SO_4}, OC, BC                          & F           & 1.89x2.50                  & \cite{seland2019, kirkevag2018production}               \\
  CanESM5    & CMIP6     & D, SS, DMS                         & C: \chem{SO_2}, OC, BC              & F           & 2.77x2.81                  & \cite{gmd-12-4823-2019}                                             \\
  CESM2      & CMIP6     & D, SS, DMS\textsubscript{clim}                          & C: \chem{SO_2}, OC, BC                           & F           & 0.94x1.25                  &  \cite{Danabasoglu2019, Tilmes2019}                                                                   \\
  IPSL-CM6A  & CMIP6     & D, SS, DMS\textsubscript{clim}               &  *C:\chem{SO_2}, BC, OC, \chem{NH_3}                          & fSST           & 2.50$\times$1.27                  &         \cite{lurton2019}                                                      \\ \bottomrule
 \end{tabularx}

 \belowtable{
  Anthropogenic emissions (C=CMIP6-CEDS, O=other, *C=CMIP6 modified);
  Interactive natural emissions (D=dust, SS=sea salt, O=biogenic organic, V=volcanic, Oce=oceanic, Veg=vegetation, DMS=dimethyl sulfide, DMS\textsubscript{clim}=dimethyl sulfide from climatology, VOC=volatile organic compounds, MSA=methane sulfonate;
  Meteorology (S=prescribed varying meteorology CTM, N=GCM nudged to analysed meteorology, fSST=fixed SST/SIC monthly fields GCM not nudged, F= free running coupled GCM, RA=combined reanalysis of meteorology and composition)
 }
 \label{table:models}
\end{sidewaystable}

\clearpage
\begin{table}
\caption{Observational mean values for the year 2000, the reference year used for computing relative trends. Each value is extracted as the intercept of the linear trend computed in the 2000-2014 period, except for $\sigma_{sp}$ and $\sigma_{ap}$, where the trends have been computed over 2000-2018. Because the required minimum number of yearly averages was set to seven, no trend could be computed in the southern African region.}
 \begin{tabular}{lcccccc}
  \tophline
                     & EUROPE & NAMERICA & SAMERICA & NAFRICA & ASIA & AUSTRALIA \\
  \middlehline
  AOD                & 0.17   & 0.10     & 0.15     & 0.26    & 0.37 & 0.10      \\
  \chem{AOD_{f}}            & 0.15   & 0.08     & 0.11     & 0.11    & 0.22 & 0.04      \\
  \chem{AOD_{c}}            & 0.03   & 0.02     & 0.03     & 0.10    & 0.09 & 0.03      \\
  AE                 & 1.43   & 1.48     & 1.26     & 0.70    & 1.16 & 1.00      \\
  \chem{PM_{2.5}} (\unit{µg\,m^{-3}})     & 12.3   & 6.9      & -        & 9.0       & -    & -         \\
  \chem{PM_{10}} (\unit{µg\,m^{-3}})      & 16.8   & 12.4     & -        & 19.7    & -    & -         \\
  \chem{SO_4} (\unit{µg\,m^{-3}})       & 2.01   & 1.45     & -        & 2.98    & 1.97 & -         \\
  $\sigma_{sp}$ (\unit{Mm^{-1}}) & 34.4   & 23.4     & -        & -       & -    & -         \\
  $\sigma_{ap}$ (\unit{Mm^{-1}})  & 6.3    & 2.6      & -        & -       & -    & -         \\
  \bottomhline
 \end{tabular}
 \label{table:obs_2000mean}
\end{table}

\cleapage
\begin{table}
\caption{Global means and trends of aerosol parameters using NorESM2 model data. The value in parenthesis is obtained by aggregating only grid-points where observation stations are located while using the complete model time series. The relative trends are calculated by averaging the absolute trends within the considered grid-points and normalizing it to the global mean for the year 2000.}
 \begin{tabular}{lcc}
  \tophline
                                & $Mean_{2000}$ & Trend (\%/yr) \\
  \middlehline
  AOD                           & (0.16) 0.14   & (+0.1) +0.2   \\
  \chem{AOD_{f}}                       & (0.09) 0.05   & (+0.4) +0.6   \\
  \chem{AOD_{c}}                       & (0.06) 0.09   & (-0.2) +0.1   \\
  AE                            & (0.78) 0.43   & (+0.2) +0.3   \\
  \chem{PM_{2.5}} (\unit{µg\,m^{-3}}) & (6.3) 9.1    & (-1.1) +0.2   \\
  \chem{PM_{10}} (\unit{µg\,m^{-3}})  & (11.7) 18.7   & (-0.6) +0.1   \\
  \chem{SO_4} (\unit{µg\,m^{-3}})   & (1.86) 0.64   & (-3.9) +0.4   \\
  $\sigma_{sp}$ (\unit{Mm^{-1}})  & (13.9) 21.2   & (-0.3) +0.2   \\
  $\sigma_{ap}$ (\unit{Mm^{-1}})   & (2.0) 0.9     & (+1.4) +1.5   \\
  \bottomhline
 \end{tabular}

 \label{table:global_trends}
\end{table}


%%FIGURES
\clearpage
\begin{figure}
 \centering
 \includegraphics[width=12cm]{../scripts/figs/hist_runs.png}
 \caption{Global AOD computed from model historical runs (OsloCTM3, GFDL-AM4, CanESM5, CESM2, IPSL-CM6A, ECHAM-HAM) at monthly (gray lines) and yearly resolutions (black lines), overlayed with the number of active observation sites in the AERONET sunphotometer network.}
 \label{fig:hist_runs}
\end{figure}

\clearpage
\begin{figure}
 \includegraphics[width=12cm]{../scripts/figs/maps/av_obs.png}
 \caption{Distribution of the observations within the different regions considered in this study. The numbers reported within each region correspond to the maximum number of stations given for the observation networks corresponding to the five observation types found in the legend.}
 \label{fig:map_obs}
\end{figure}

\clearpage
\begin{figure}
 \includegraphics[width=16cm]{../scripts/figs/ts/panel-od550aer.png}
 \caption{Regional time series of AOD. The dark blue line corresponds to the median and the light blue envelope is bound by the first and third quartiles of all valid points at the corresponding month, respectively. The blue dots correspond to the yearly averages which are used to compute the linear trend. The latter is displayed as a continuous line when the trend is significant and as a dashed line when it is not. Trend values, an error estimate and significance value are given in each pane.}
 \label{fig:ts_aod}
\end{figure}

\clearpage
\begin{figure}[t]
 \includegraphics[width=16cm]{../scripts/figs/representativity-od550aer-lines.png}
 \caption{Three regional AOD time series and respective trends,  constructed from model data (NorESM2) for the investigation of the representativity of the observational data. The upper figures correspond to the number of points used to compute the regional time series for the three different datasets. The lower figures show the time series, the trends, and the resulting representativity value (black, bold). The blue color ($Ref_{time}$) corresponds to the model output collocated in space and time with the available observations. The upper graphs show an overall increase in the number of available observations (more stations) combined with a seasonal cycle (less AOD available in wintertime). The orange color ($Exp_{time}/Ref_{space}$) corresponds to the model output collocated in space to the stations providing measurements, using the complete time series from 2000 to 2014. The green color ($Exp_{space}$) corresponds to the model output in the whole geographic region (see \ref{fig:map_obs}), using all of the grid boxes without any collocation to the observations.}
 \label{fig:representativity}
\end{figure}

\clearpage
\begin{figure}[t]
 \includegraphics[width=12cm]{../scripts/figs/heatmaps/OBS.png}
 \caption{Regional trends of the aerosol properties computed with the observation datasets. The color of the circles corresponds to the slope, while the radius indicates the p-value. The largest circles represent the trends significant with a confidence level of 95\%. The circles bordered with a black line indicate the trends associated with a representativity greater than 50\%.}
 \label{fig:obs_trends}
\end{figure}

\clearpage
\begin{figure}[t]
 \includegraphics[width=16cm]{../scripts/figs/heatmaps/BARS.png}
 \caption{Regional trends of the aerosol properties computed with observations and models collocated in space and time to the observations. The error bars correspond to the uncertainty of the trend as calculated using both the uncertainty in the Theil−Sen slope and the residuals. The bold font indicates that the trends are significant at a confidence level of 95\% (p-val<0.05).}
 \label{fig:bars}
\end{figure}

\clearpage
\begin{figure}[t]
 \includegraphics[width=16cm]{../scripts/figs/trends_map2.png}
 \caption{Global trends of aerosol properties using NorESM2 data regridded at a 5x5 degrees resolution. The blue and red dots dots indicate significant negative and positive trends, respectively.}
 \label{fig:global_trends}
\end{figure}

\clearpage
\begin{figure}[t]
 \includegraphics[width=16cm]{../scripts/figs/abs_species_trends.png}
 %\includegraphics[width=16cm]{../scripts/figs/pannel-abs_species_trends.png}
 \caption{Absolute trends in OD and emissions of the main aerosol species computed with NorESM2. The y-axis of the trends in OD and the emissions is given according to the power of 10 indicated at the top left corner of each of the subplots.}
 \label{fig:species}
\end{figure}





%% When figures and tables are placed at the end of the MS (article in one-column style), please add \clearpage
%% between bibliography and first table and/or figure as well as between each table and/or figure.


%% ONE-COLUMN FIGURES

%%f
%\begin{figure}[t]
%\includegraphics[width=12cm]{FILE NAME}
%\caption{TEXT}
%\end{figure}
%
%%% TWO-COLUMN FIGURES
%
%%f
%\begin{figure*}[t]
%\includegraphics[width=16cm]{FILE NAME}
%\caption{TEXT}
%\end{figure*}
%
%
%%% TABLES
%%%
%%% The different columns must be seperated with a & command and should
%%% end with \\ to identify the column brake.
%
%%% ONE-COLUMN TABLE
%
%%t
%\begin{table}[t]
%\caption{TEXT}
%\begin{tabular}{column = lcr}
%\tophline
%
%\middlehline
%
%\bottomhline
%\end{tabular}
%\belowtable{} % Table Footnotes
%\end{table}
%
%%% TWO-COLUMN TABLE
%
%%t
%\begin{table*}[t]
%\caption{TEXT}
%\begin{tabular}{column = lcr}
%\tophline
%
%\middlehline
%
%\bottomhline
%\end{tabular}
%\belowtable{} % Table Footnotes
%\end{table*}
%
%%% LANDSCAPE TABLE
%
%%t
%\begin{sidewaystable*}[t]
%\caption{TEXT}
%\begin{tabular}{column = lcr}
%\tophline
%
%\middlehline
%
%\bottomhline
%\end{tabular}
%\belowtable{} % Table Footnotes
%\end{sidewaystable*}
%
%
%%% MATHEMATICAL EXPRESSIONS
%
%%% All papers typeset by Copernicus Publications follow the math typesetting regulations
%%% given by the IUPAC Green Book (IUPAC: Quantities, Units and Symbols in Physical Chemistry,
%%% 2nd Edn., Blackwell Science, available at: http://old.iupac.org/publications/books/gbook/green_book_2ed.pdf, 1993).
%%%
%%% Physical quantities/variables are typeset in italic font (t for time, T for Temperature)
%%% Indices which are not defined are typeset in italic font (x, y, z, a, b, c)
%%% Items/objects which are defined are typeset in roman font (Car A, Car B)
%%% Descriptions/specifications which are defined by itself are typeset in roman font (abs, rel, ref, tot, net, ice)
%%% Abbreviations from 2 letters are typeset in roman font (RH, LAI)
%%% Vectors are identified in bold italic font using \vec{x}
%%% Matrices are identified in bold roman font
%%% Multiplication signs are typeset using the LaTeX commands \times (for vector products, grids, and exponential notations) or \cdot
%%% The character * should not be applied as mutliplication sign
%
%
%%% EQUATIONS
%
%%% Single-row equation
%
%\begin{equation}
%
%\end{equation}
%
%%% Multiline equation
%
%\begin{align}
%& 3 + 5 = 8\\
%& 3 + 5 = 8\\
%& 3 + 5 = 8
%\end{align}
%
%
%%% MATRICES
%
%\begin{matrix}
%x & y & z\\
%x & y & z\\
%x & y & z\\
%\end{matrix}
%
%
%%% ALGORITHM
%
%\begin{algorithm}
%\caption{...}
%\label{a1}
%\begin{algorithmic}
%...
%\end{algorithmic}
%\end{algorithm}
%
%
%%% CHEMICAL FORMULAS AND REACTIONS
%
%%% For formulas embedded in the text, please use \chem{}
%
%%% The reaction environment creates labels including the letter R, i.e. (R1), (R2), etc.
%
%\begin{reaction}
%%% \rightarrow should be used for normal (one-way) chemical reactions
%%% \rightleftharpoons should be used for equilibria
%%% \leftrightarrow should be used for resonance structures
%\end{reaction}
%
%
%%% PHYSICAL UNITS
%%%
%%% Please use \unit{} and apply the exponential notation


\end{document}
